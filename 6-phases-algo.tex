\documentclass[11pt]{article}

\usepackage[margin=1in]{geometry}
\usepackage{amsmath,amssymb,amsfonts}
\usepackage{xcolor}
\usepackage{amsthm}
\usepackage{mathtools}
\usepackage[]{hyperref}
\hypersetup{
  colorlinks=false,
}
\usepackage{cleveref}

\usepackage{tlatex}
\usepackage{color}
\definecolor{boxshade}{gray}{.85}
\setboolean{shading}{true}

\newtheorem{property}{Property}
\newtheorem{claim}{Claim}
\newtheorem{rrule}{Rule}

\title{An Unauthenticated BFT Consensus Algorithm in 6 Phases}
\date{September 2021}
\author{Giuliano Losa}

\begin{document}

\maketitle

\section{Introduction}


This document describes a new BFT consensus algorithm in the unauthenticated Byzantine model.

This algorithm is live under eventual synchrony, meaning that, assuming quorum intersection and quorum availability, the malicious nodes cannot prevent well-behaved nodes from reaching a decision.
In contrast, in SCP, malicious nodes can delay a decision forever, and liveness can only be guaranteed if we assume that malicious nodes are eventually evicted from the slices of well-behaved nodes.


Like SCP, nodes following the algorithm use an amount of space that is bounded by
a constant that is independent of the length of the execution.

The algorithm has 6 phases leading to a decision: suggest/proof; propose;
vote-1; vote-2; vote-3; vote-4. 
%A leader proposal is decided in 4 phases in the best case, which matches the Stellar Consensus Protocol.

The key to the liveness guarantee is that the leader determines whether a value v is safe at a round when the vote-2 messages of a blocking set show that v is safe, while follower nodes determine that v is safe when the vote-1 messages of a blocking set show that v is safe.
Thus, any value determined safe by a leader is also determined safe by all other well-behaved nodes.

Information-Theoretic HotStuff (IT-HS), which seems to be the state-of-the-art
unauthenticated BFT consensus algorithm, uses 7 phases: suggest/proof; propose; echo;
key-1; key-2; key-3; commit; decision.

The key to obtaining a 6 phases is to not rely on locks, contrarily to IT-HS, and
instead require nodes to check that the leader's value is safe before doing
anything.  This way we know that as soon as f+1 endorse a value, then the value
is safe.  In IT-HS, a node that has no lock may endorse an unsafe value from a
Byzantine leader, and thus an additional phase (the echo phase) is needed to
first establish that the leader's value is safe.

\section{The Algorithm}

We assume that there are at most $f$ malicious nodes out of $n$ nodes with $3f<n$.
Any set of $2f+1$ nodes or more is called a quorum, and any set of $f+1$ nodes or more is called a blocking set.

We describe the algorithm in a high-level round-by-round model. Each node executes a
sequence of rounds, starting at round 0. Rounds are communication-closed and
each round has a unique, predetermined leader. We assume that, before round
GSR, the network is unreliable, i.e. messages can be lost; during and after
GSR, we assume that every message sent by a well-behaved node is received.

\subsection{Messages}

A node can send the following messages.
\begin{itemize}
  \item \textit{proposal}, \textit{vote-1}, \textit{vote-2}, \textit{vote-3},
    \textit{vote-4}, each containing a round and a value.
  \item A \textit{suggest} message, containing
    \begin{enumerate}
      \item the highest \textit{vote-2} message that the node sent and
      \item the second highest round for which the node sent a \textit{vote-2}
        message, noted \textit{prev-vote-2}, and
      \item the highest \textit{vote-3} message that the node sent.
    \end{enumerate}
  \item A \textit{proof} message, containing
    \begin{enumerate}
      \item the highest \textit{vote-1} message that the node sent and
      \item the second highest round for which the node sent a \textit{vote-1}
        message, noted \textit{prev-vote-1}, and
      \item the highest \textit{vote-4} message that the node sent.
    \end{enumerate}
\end{itemize}

\subsection{State}

Across rounds, a node only has to remember the highest vote-1 and vote-2,
vote-3 and vote-4 message it sent, and the round of the second highest vote-1
and vote-2 messages it sent.

\subsection{Evolution of a round}

A round \texttt{r} proceeds as follows:
\begin{enumerate}
  \item Upon starting the round, a node \texttt{n} does the following:
    \begin{enumerate}
      \item it broadcasts a \textit{proof} message for the current round and
      \item it sends a \textit{suggest} message to the leader of the round.
    \end{enumerate}
  \item When the leader has determined that value \texttt{v} is safe to propose
    in current round (as described in \Cref{rule:picking-safe-proposal}), it broadcasts a
    \textit{proposal} message for the current round and for \texttt{v}.
  \item When a node determines that the leader's proposal is safe in the
    current round (as described in \Cref{rule:checking-safe-proposal}), it broadcasts a vote-1
    message for the current round and the leader's proposal.
    \item A node that receives a quorum of \textit{vote-1} messages for the
      current round and for the same value \texttt{v} sends a \textit{vote-2}
      message for the current round and for \texttt{v}.
    \item A node that receives a quorum of \textit{vote-2} messages for the
      current round and for the same value \texttt{v} sends a \textit{vote-3}
      message for the current round and for \texttt{v}.
    \item A node that receives a quorum of \textit{vote-3} messages for the
      current round and for the same value \texttt{v} sends a \textit{vote-4}
      message for the current round and for \texttt{v}.
    \item A node that receives a quorum of \textit{vote-4} messages for the
      current round and for the same value \texttt{v} decides \texttt{v}
\end{enumerate}

NOTE: in an FBQS, information must be propagated throughout the
system using the cascading mechanism (e.g. if a blocking set sent
\texttt{vote-2} then send it too). Information in \texttt{proof} and
\texttt{suggest} messages may also need to be propagated.

\begin{rrule}
\label{rule:picking-safe-proposal}
All values are safe in round 0. If \texttt{r}$\neq 0$, a leader determines that the value \texttt{v} is safe to propose in round
  \texttt{r} when the following holds:
  \begin{enumerate}
    \item A quorum \texttt{q} has sent \textit{suggest} messages in round \texttt{r}, and
    \item According to what is reported in \textit{suggest} messages, either
      \begin{enumerate}
        % \item the round is 0, or
          % \label{case:round_0_leader}
        \item no member of \texttt{q} sent any \textit{vote-3} before round
          \label{case:no_vote_leader}
          \texttt{r}, or
        \item there is a round $\texttt{r'}< \texttt{r}$ such that
          \label{case:highest_vote_leader}
          \begin{enumerate}
            \item no member of \texttt{q} sent any \textit{vote-3} messages for a
              round strictly higher than \texttt{r'}, and
            \item any member of \texttt{q} that sent a \textit{vote-3} message in
              round \texttt{r'} did so with value \texttt{v}, and
            \item there is a blocking set \texttt{b} (e.g. $ \texttt{f} +1$
              nodes) that all claim in their \textit{suggest} messages that
              \texttt{v} is safe at \texttt{r'} (see \Cref{rule:claims_safe_leader}).
          \end{enumerate}
      \end{enumerate}
  \end{enumerate}
\end{rrule}

\begin{rrule}
  \label{rule:claims_safe_leader}
We say that a node claims that \texttt{v} is safe in \texttt{r'} in a
suggest message when either
\begin{enumerate}
  \item \texttt{r'} is 0,
  \item the node's highest \textit{vote-2} message, as reported in the
    \textit{suggest} message, was sent at round $\texttt{r''}\geq \texttt{r'}$
    and for value \texttt{v}, or
  \item the second highest round for which the node sent a \textit{vote-2}
    message, as reported in the suggest message, is a round $\texttt{r''}\geq
    \texttt{r'}$.
\end{enumerate}
\end{rrule}

A node that receives a proposal from the leader of the current round determines
that the value \texttt{v} is safe in round $\texttt{r}\neq 0$ like the
leader does, except that it uses \textit{proof} messages instead of
\textit{suggest} message, \textit{vote-4} instead of \textit{vote-3}, and
\textit{vote-1} instead of \textit{vote-2}:

\begin{rrule}
\label{rule:checking-safe-proposal}

A node that receives a proposal from the leader of the current round determines
that the value \texttt{v} is safe to propose in round
\texttt{r} when:
\begin{enumerate}
  \item A quorum \texttt{q} has sent \textit{proof} messages in round \texttt{r}, and
  \item According to what is reported in \textit{proof} messages, either
    \begin{enumerate}
      % \item the round is 0, or
        % \label{case:round_0}
      \item no member of \texttt{q} sent any \textit{vote-4} before round
        \texttt{r}, or
        \label{case:no_votes}
      \item there is a round $\texttt{r'}< \texttt{r}$ such that
        \label{case:highest_vote}
        \begin{enumerate}
          \item no member of \texttt{q} sent any \textit{vote-4} messages for a
            round strictly higher than \texttt{r'}, and
          \item any member of \texttt{q} that sent a \textit{vote-4} message in
            round \texttt{r'} did so with value \texttt{v}, and
          \item there is a blocking set \texttt{b} (e.g. $ \texttt{f} +1$
            nodes) that all claim in their \textit{suggest} messages that
            \texttt{v} is safe at \texttt{r'} (as described in \Cref{rule:claims_safe})
        \end{enumerate}
    \end{enumerate}
\end{enumerate}
\end{rrule}

\begin{rrule}
  \label{rule:claims_safe}
We say that a node claims that \texttt{v} is safe in \texttt{r'} in a
\textit{proof} message when either
\begin{enumerate}
  \item \texttt{r'} is 0,
  \item the node's highest \textit{vote-1} message, as reported in the
    \textit{proof} message, was sent at round $\texttt{r''}\geq \texttt{r'}$
    and for value \texttt{v}, or
  \item the second highest round for which the node sent a \textit{vote-1}
    message, as reported in the proof message, is a round $\texttt{r''}\geq
    \texttt{r'}$.
\end{enumerate}
\end{rrule}


\section{Liveness argument}

Consider a round \texttt{r} that a) has a well-behaved leader and b) is
sufficiently long for all the messages sent by well-behaved nodes to
well-behaved nodes to be received.

\begin{claim}
  \label{claim:if_safe_then_termination}
  If all well-behaved nodes determine that the leader's value is safe, then a
  decision is made by all well-behaved nodes.
\end{claim}

\begin{claim}
  \label{claim:blocking_claims_safe}
  If a well-behaved node claims that \texttt{v} is safe at \texttt{r'} in its suggest message in round $\texttt{r}> \texttt{r'}$, then there is a blocking set \texttt{b} composed entirely of well-behaved nodes such that, in any \textit{proof} message in rounds greater or equal to \texttt{r}, every member of \texttt{b} claims that \texttt{v} is safe at \texttt{r'}.
\end{claim}
\begin{proof}
  This rests on the fact that what is claimed safe by a well-behaved node can only increase.
\end{proof}

\begin{claim}
  \label{claim:proposal_determined_safe}
  A well-behaved node eventually determines that the leader's value \texttt{v} is safe.
\end{claim}
\begin{proof}
  According to the rule that the leader uses to propose a value (\cref{rule:picking-safe-proposal}), there are three cases. First, if the round is 0 then all well-behaved nodes trivially determine that the leader's value is safe. 

  Second (\Cref{case:no_vote_leader}), suppose that the round is not 0 and that the leader proposes \texttt{v} because a quorum \texttt{q} reports not sending any \texttt{vote-3} messages. Then, there is an entirely well-behaved blocking set \texttt{b} that never sent any \texttt{vote-3} messages. Since \texttt{vote-4} messages are sent in response to a quorum of \texttt{vote-3} messages, and since a quorum and a blocking set must have a well-behaved node in common, we conclude that no well-behaved node ever sent a \texttt{vote-4} message. Thus, once a well-behaved node \texttt{n} receives \textit{proof} messages from all other well-behaved nodes, \texttt{n} concludes that the proposal is safe according to \Cref{case:no_votes} of \Cref{rule:checking-safe-proposal}.

  Third (\Cref{case:highest_vote_leader}), suppose that the round is not 0 and we have a quorum \texttt{q} and a round $\texttt{r'}< \texttt{r}$ such that:
  \begin{enumerate}
    \item no member of \texttt{q} sent any \textit{vote-3} messages for a
      round strictly higher than \texttt{r'}, and
      \label{item:one}
    \item any member of \texttt{q} that sent a \textit{vote-3} message in
      round \texttt{r'} did so with value \texttt{v}, and
      \label{item:two}
    \item there is a blocking set \texttt{b} (e.g. $ \texttt{f} +1$
      nodes) that all claim in their \textit{suggest} messages that
      \texttt{v} is safe at \texttt{r'} (see \Cref{rule:claims_safe_leader}).
      \label{item:three}
  \end{enumerate}

  We make the following observations:
  \begin{enumerate}
    \item[a)] By \Cref{item:one} above, no well-behaved node sent any \textit{vote-4} message in any round higher than \texttt{r'}; otherwise, a quorum would have sent the corresponding \texttt{vote-3} messages and, by the quorum-intersection property, this contradicts \Cref{item:one}.
    \item[b)] By \Cref{item:two} above, for a similar reason, any well-behaved node that sent a \texttt{vote-4} message in round \texttt{r'} did so for value \texttt{v}.
    \item[c)] By \Cref{item:three}, there is a well-behaved node that claims that \texttt{v} is safe in \texttt{r'} in its \textit{suggest} message. By \Cref{claim:blocking_claims_safe}, we conclude that there is a blocking set \texttt{b} composed entirely of well-behaved nodes that claim in their \textit{proof} messages that \texttt{v} is safe at \texttt{r'}.
  \end{enumerate}
  By Items a), b), and c) and \Cref{rule:checking-safe-proposal}, we conclude that, once every well-behaved node has received a \textit{proof} message from every other well-behaved node, every well-behaved node determines that the leader's proposal is safe.
\end{proof}

Finally, by \Cref{claim:proposal_determined_safe} and
\Cref{claim:if_safe_then_termination}, we conclude that a decision is reached
by all well-behaved nodes.

\end{document}
